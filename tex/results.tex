\subsection{Calculation results}

\begin{figure}[H]
    \centering
    \includegraphics[width=0.8\textwidth]{traces}
    \caption{The traces of the \(100\)th to \(1000\)th particles from \(t = 390\) to \(t = 500\).}
    \label{fig:traces}
\end{figure}

We select several particles (\(100\)th, \(200\)th, \(\ldots\), \(1000\)th) and plot their traces
from \(t = 390\) to \(t = 500\) (i.e., at \(T \approx 1.073\)) in Figure~\ref{fig:traces}.
The figure shows that these particles are wandering around a certain
range. The straight lines passing through the box does not mean they just at a very
large speed, since after thermalization, the speed of particles would not change so
much. These lines just indicate that these particles move out the current cell,
and their images reappear from the other side of the cell, as required by the PBCs.

Now let us calculate the \ce{Ar} system's thermodynamic properties.
The pressure is calculated from the virial theorem, as shown in~\eqref{eq:P}
%
\begin{equation}\label{eq:P}
    \frac{ P }{ \rho k_B T } = 1 - \frac{ 1 }{ 3N k_B T } \sum_{i=1}^{N}
    \sum_{\alpha=1}^{3}
    \biggl \langle r_i^\alpha \frac{ \partial u }{ \partial r_i^\alpha } \biggr \rangle,
\end{equation}
%
where \(\langle \cdot \rangle\) denotes the time or ensemble average,
and \(i\) labels all the particles, and \(\alpha\) labels the three spatial dimensions.
Since the virial part is derived from \(u\), so the second term we get from our code
is already in unit of \(k_B\). Theferfore, the actual code is
equivalent to
%
\begin{equation}\label{eq:P}
    \frac{ \beta P }{ \rho } = 1 + \frac{ 1 }{ 3N T } \sum_{i=1}^{N}
    \langle \bm{r}_i \cdot \bm{F}_i \rangle,
\end{equation}
%
where \(\beta = \nicefrac{ 1 }{ k_B T }\) in Verlet's paper\cite{Verlet},
which is \(0.90\) at \(\rho = 0.75\) and \(T = 1.069\).

We pick steps from \(15000\) to the last step (total \(9046\) steps) and do the time average,
what we get is \(\nicefrac{ \beta P }{ \rho } \approx 1.0002\), which is quite
different from \(0.90\).

The Maxwell velocity distribution is
%
\begin{equation}
    f(v) = \biggl( \frac{m}{2\pi k_B T} \biggr)^{3/2}
    4\pi v^2 \exp{\biggl( -\frac{m v^2}{2 k_B T} \biggr)},
\end{equation}
%
where \(f(v)\) is a probability distribution function, and \(v\) is the magnitude of the
velocity vector.
We plot histograms of velocity distribution of the particles after
\(500\), \(1000\), \(2000\), \(3000\), \(5000\), \(8000\), \(10000\), \(12000\), \(16000\) steps,
respectively, in Figure~\ref{fig:maxwell}.
%
\begin{figure}
    \centering
    \includegraphics[width=\textwidth]{maxwell}
    \caption{Histograms of velocity distribution of the particles after
        \(500\), \(1000\), \(2000\), \(3000\), \(5000\), \(8000\), \(10000\), \(12000\), \(16000\) steps.}
    \label{fig:maxwell}
\end{figure}
%
We can see that at the beginning of the computation, the velocities of the particles
seem to be random, but as the number of time steps increases, a pattern starts to
emerge. After the \(5000\)th step (\(t = 160\)), we can almost see a Maxwell distribution.
From the \(10000\)th (\(t = 320\)) to the \(16000\)th (\(t = 512\)) steps, the pattern is
already Maxwellian and does not change so much.
