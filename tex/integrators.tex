\subsection{Integrators}

There exist many algorithms for integrating ordinary differential equations.
In this section, we consider the particular case of numerically integrating the equations of
motion for a dynamical system described by a time-independent Hamiltonian, i.e., the
classical many-particle system as a microcanonical ensemble.
Here, we use the velocity Verlet algorithm as our integrator.
The mathematical expression of the algorithm is shown in~\eqref{eq:vvr} and~\eqref{eq:vvv}.
%
\begin{align}
    \bm{r}(t + \Delta t) & = \bm{r}(t) + \bm{v}(t) \Delta t + \frac{ \bm{a}(t) }{ 2 } \Delta t^2,\label{eq:vvr} \\
    \bm{v}(t + \Delta t) & = \bm{v}(t) + \frac{ \bm{a}(t) + \bm{a}(t + \Delta t) }{ 2 } \Delta t,\label{eq:vvv}
\end{align}
%
where $\Delta t$ is the length of each time step, $\bm{r}(t + \Delta t)$,
$\bm{v}(t + \Delta t)$, and $\bm{a}(t + \Delta t)$
are the coordinates, velocity, and acceleration of the particle at time $t + \Delta t$.
In our dimensionless system, the acceleration and the force have the same value,
where the method of calculating forces is described in section~\ref{ssec:force}.
The force at time $t$ is calculated by all the positions of the particles at time $t$.

The standard implementation scheme of the velocity Verlet algorithm is:
%
\begin{enumerate}
    \item Calculate $\bm{v}(t + \nicefrac{\Delta t}{2}) = \bm{v}(t) + \frac{1}{2} \bm{a}(t) \Delta t$;\label{it:vv1}
    \item Calculate $\bm{r}(t + \Delta t) = \bm{r}(t) + \bm{v}(t + \nicefrac{\Delta t}{2}) \Delta t$;\label{it:vv2}
    \item Derive $\bm{a}(t + \Delta t)$ from the Lennard--Jones potential using $\bm{r}(t + \Delta t)$;\label{it:vv3}
    \item Calculate $\bm{v}(t + \Delta t) = \bm{v}(t + \nicefrac{\Delta t}{2}) + \frac{1}{2} \bm{a}(t + \Delta t) \Delta t$.\label{it:vv4}
\end{enumerate}
%
The corresponding code for the above algorithm is shown in Snippet~\ref{lst:take_one_step},
defined by function \code{take_one_step!}.
%
\begin{algorithm}
    \caption{Move all particles one step forward in the simulation cell with time step
        $\Delta t$, using the velocity Verlet integrator.}
    \label{lst:take_one_step}
    \begin{juliacode}
        function take_one_step!(particles, box::Box, Δt, ::VelocityVerlet)
            for (particle, 𝐟) in zip(particles, force(particles, box))
                particle.velocity += 𝐟 * Δt / 2  # 𝐯(t + Δt / 2)
                particle.coordinates += particle.velocity * Δt  # 𝐫(t + Δt)
                map!(Base.Fix2(mod, box.side_length), particle.coordinates, particle.coordinates)  # Move `𝐫` back to `0 - L` range
            end
            for particle in particles
                𝐟 = force(particle, particles, box)  # 𝐚(t + Δt)
                particle.velocity += 𝐟 * Δt / 2  # 𝐯(t + Δt)
            end
            return particles
        end
    \end{juliacode}
\end{algorithm}
%
The input \code{particles} and \code{box} are the simulation particles and the cell we
are interested in, with its number density to be $0.75$ here. And each time step
$\Delta t$ is $0.032$ (which Verlet used in his simulation). For \ce{Ar}, it corresponds
to the real world time of
%
\begin{equation}
    t_\textnormal{real} = t \sqrt{\frac{ m \sigma^2 }{ 48 \varepsilon }}
    \approx 0.032 \times \num{3.112e-13}
    \approx \qty{9.96e-15}{\second},
\end{equation}
%
which is roughly \qty{1e-14}{\second}, as claimed by Verlet\cite{Verlet}.
We should notice that the time step cannot be too large, which will result in force
increasing rapidly, and within each time step, the particles can come too close,
causing our simulation hard to converge.
Note that we have two \code{for}-loops here. The first \code{for}-loop first calculates
the forces each particle feels using \code{force(particles, box)}, then it updates
the velocity and position of the particles iteratively, as explained in steps~\ref{it:vv1}
and~\ref{it:vv2}.
The last line in that loop is how we apply the PBCs on the position of each particle,
as discussed in section~\ref{ssec:pbcs}.
Here, since the cell has size $L \times L \times L$, where $L$ is a number larger than
$0$, we need to do a modulo operation against $L$.

However, one thing to be very careful about is that since the \code{for}-loop updates
the position of each particle one by one, we cannot do force calculations before
step~\ref{it:vv1} in this loop,
since when we compute the force on the second particle at time $t$, the first particle
have already moved from its original position at time $t$, and we are computing the force
using the position of the first particle at time $t + \Delta t$ since step~\ref{it:vv2}
has already been performed.
That is why we have to calculate the forces at time $t$ before step~\ref{it:vv1} and
why we have to use two \code{for}-loops.
In languages where loops are slow, like Python, people are more accustomed to use
vectorized operations so all the postions of the particles are updated at the same time
without any problem.
But in Julia, loops are already fast, so using high-dimensional arrays are not needed,
but we have to act with caution so that we do not introduce unexpected errors.

\begin{algorithm}
    \caption{Function \code{take_n_steps!} runs the simulation for $n$ consecutive steps.}
    \label{lst:take_n_steps}
    \begin{juliacode}
        function take_n_steps!(particles, box::Box, n, Δt, ::VelocityVerlet)
            @showprogress for _ in 1:n
                take_one_step!(particles, box, Δt, VelocityVerlet())
            end
            return particles
        end
    \end{juliacode}
\end{algorithm}

In the second \code{for} loop, since we have already performed steps~\ref{it:vv1}
and~\ref{it:vv2}, we can safely calculate the forces on each particle and hence update its
coordinates, as explained in steps~\ref{it:vv3} and~\ref{it:vv4}.
However, in the first implementation of my code, I made a mistake: for any particle, say $a$,
when calculating the force exerted on it at time $t + \Delta t$, I used the coordinates of
all other particles at time $t$ while using the coordinates of $a$ at time $t + \Delta t$.
That is, when the force of other particles already sends $a$ to a new position after
step~\ref{it:vv2}, in step~\ref{it:vv3} I applied a force that is going to drag it
back to its original position. This mistake caused the energy blows up at a rapid
speed as a function of time, and reaching thermal equilibrium requires tens of thousands
steps since we somewhat cancel the force back and forth, and we had to take a very small
time step (around $10^{-4}$) to make the simulation run. This took me a long time to debug.

In a simulation, we certainly need to run more than one step. The code is shown in
Snippet~\ref{lst:take_n_steps}.
Since it may take a long time to run $n$ steps, we add a progress bar showing the
progress during the simulation, using the \code{@showprogress} macro.
