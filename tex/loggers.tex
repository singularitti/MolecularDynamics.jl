\subsection{Loggers}

To track the positions and velocities of all the particles, we need some loggers.
Here we define two types, a \code{Step} type which represents each step of the molecular
dynamics, and a \code{Logger} type which stores all the \code{Step}s we have run,
as shown in Snippet~\ref{lst:logger}.
%
\begin{algorithm}
    \caption{The \code{Step} and \code{Logger} types which tracks each step of the
        molecular dynamics.}
    \label{lst:logger}
    \begin{juliacode}
        struct Step{N}
            Δt::Float64
            snapshot::SVector{N,Particle}
        end

        struct Logger{N}
            history::ElasticVector{Step{N}}
        end
    \end{juliacode}
\end{algorithm}
%
To track and continue our simulation, we also need to add two new methods to function
\code{take_n_steps!}, as shown in Snippet~\ref{lst:trackncontinue}.
%
\begin{algorithm}
    \caption{The \code{Step} and \code{Logger} types which tracks each step of the
        molecular dynamics.}
    \label{lst:trackncontinue}
    \begin{juliacode}
        function take_n_steps!(
            logger::Logger{N}, particles, box::Box, n, Δt, ::VelocityVerlet
        ) where {N}
            @showprogress for _ in 1:n
                take_one_step!(particles, box, Δt, VelocityVerlet())
                push!(logger.history, Step(Δt, SVector{N}(deepcopy(particles))))
            end
            return particles
        end
        function take_n_steps!(logger::Logger, box::Box, n, Δt, ::VelocityVerlet)
            particles = logger.history[end].snapshot
            return take_n_steps!(logger, particles, box, n, Δt, VelocityVerlet())
        end
    \end{juliacode}
\end{algorithm}
%
In the first method, all steps are saved in variable \code{logger} if we start a new
molecular dynamics simulation, and the second method indicates we are restarting an
existing simulation with all previous steps from the variable \code{logger}.

We may use Julia's serialization library \code{Serialization} to
serialize (save) and deserialize (load) the logger. Or, we can use third-party packages
such as \href{https://github.com/JuliaIO/JLD.jl}{\code{JLD.jl}} and
\href{https://github.com/JuliaIO/JLD2.jl}{\code{JLD2.jl}}.
But these file formats are binary, so they are not human readable.
If we want to read these files by ourselves, serializing them into readable formats
is also necessary. Here, we adopt the JSON format, which
is an open standard and data interchange format that uses human-readable text to
store and transmit data objects. With the help of it, we can store the locations
and velocities of the particles into a JSON file. The corresponding code is shown in
Snippet~\ref{lst:json}.
%
\begin{algorithm}
    \caption{Code to save (load) the information to (from) a JSON file.}
    \label{lst:json}
    \begin{juliacode}
        using JSON: JSON

        function Base.Dict(particle::Particle)
            return Dict("coordinates" => particle.coordinates, "velocity" => particle.velocity)
        end

        function Base.print(io::IO, particle::Particle)
            JSON.print(io, Dict(particle))
            return nothing
        end
        function Base.print(io::IO, particles::AbstractVector{Particle})
            JSON.print(io, map(Dict, particles))
            return nothing
        end

        function save(filename, data)
            open(filename, "w") do io
                print(io, data)
            end
            return nothing
        end

        function load(filename)
            data = open(filename, "r") do io
                JSON.parse(io)
            end
            return data
        end
    \end{juliacode}
\end{algorithm}
