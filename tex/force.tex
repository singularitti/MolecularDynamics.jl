\subsection{Force calculation}

Now let us talk about the Lennard--Jones potential.
The original form of the two-particle potential would look like~\eqref{eq:uljunit}:
%
\begin{equation}\label{eq:uljunit}
    u(r_{ij}) = 4 \varepsilon \biggl( \Bigl( \frac{ \sigma }{ r_{ij} } \Bigr)^{12}
    -\Bigl( \frac{ \sigma }{ r_{ij} } \Bigr)^6 \biggr),
\end{equation}
%
where the parameters for \ce{Ar} are $\varepsilon = \SI{0.0104}{\electronvolt}$,
$\sigma = \SI{3.40}{\angstrom}$, and $m = \SI{39.948}{\atomicmassunit}$, where
$\si{\atomicmassunit}$ is the atomic mass unit.
In the dimensionless form,
\eqref{eq:uljunit} would look like~\eqref{eq:ulj} and the force would
look like~\eqref{eq:flj}.
%
\begin{align}
    u_{ij} = u(r_{ij}) & = 4 \biggl( \Bigl( \frac{ 1 }{ r_{ij} } \Bigr)^{12}
    -\Bigl( \frac{ 1 }{ r_{ij} } \Bigr)^6 \biggr), \label{eq:ulj}            \\
    \bm{F}_i = m\frac{ d^2 \bm{r}_i }{ d t^2 } = -\sum_{j \neq i}\frac{ \partial u_{ij} }{ \partial \bm{r}_i }
                       & = 48\sum_{j \neq i} (\bm{r}_i - \bm{r}_j)
    \biggl( \Bigl( \frac{ 1 }{ r_{ij} } \Bigr)^{14}
    -\frac{ 1 }{ 2 } \Bigl( \frac{ 1 }{ r_{ij} } \Bigr)^8 \biggr). \label{eq:flj}
\end{align}
%
And the numerical gradient of~\eqref{eq:ulj} is
%
\begin{equation}\label{eq:dudr}
    \begin{bmatrix}
        \nicefrac{ \partial u }{ \partial x } \\
        \nicefrac{ \partial u }{ \partial y } \\
        \nicefrac{ \partial u }{ \partial z }
    \end{bmatrix}
    \Rightarrow
    \begin{bmatrix}
        \nicefrac{ \bigl( u(\bm{r} + dx \hat{\bm{x}}) - u(\bm{r}) \bigr) }{ dx } \\
        \nicefrac{ \bigl( u(\bm{r} + dy \hat{\bm{y}}) - u(\bm{r}) \bigr) }{ dy } \\
        \nicefrac{ \bigl( u(\bm{r} + dz \hat{\bm{z}}) - u(\bm{r}) \bigr) }{ dz }
    \end{bmatrix}.
\end{equation}
%
I found in the notes, equation \eqref{eq:flj} is off by a factor of $48$.
Only after adding it could make \eqref{eq:dudr} and \eqref{eq:flj} match.
I suppose that this is because we basically treat
$\varepsilon$, $\sigma$, and $m$ (the mass of each particle) as $1$ here (though they
are not). So in Snippet~\ref{lst:gradient} I define the $\nabla u$ and the acceleration
particle $a$ experiences (induced only by particle $b$) as below.
%
\begin{algorithm}
    \caption{The gradient of the Lennard--Jones potential and the acceleration
        $d^2 \bm{r}_a / d t^2$. Note the factor $48$ and the negative sign.}
    \label{lst:gradient}
    \begin{juliacode}
        function potential_gradient(𝐫ᵢⱼ)
            rᵢⱼ = norm(𝐫ᵢⱼ)
            return 48𝐫ᵢⱼ * (inv(rᵢⱼ^8) / 2 - inv(rᵢⱼ^14))
        end

        function Force(a::Particle)
            return function (b::Particle)
                return Force(-potential_gradient(a.coordinates - b.coordinates))
            end
        end
    \end{juliacode}
\end{algorithm}
%
The corresponding unit test is shown in Snippet~\ref{lst:ut_gradient}.
%
\begin{algorithm}
    \caption{The unit test of function \code{potential_gradient}.}
    \label{lst:ut_gradient}
    \begin{juliacode}
        𝐫 = [4, 5, 6]  # Distance between two particles
        δ = 0.00005
        u₀ = potential_energy(𝐫)  # Potential energy between the two particles
        for i in 1:3  # Loop over each component of vector `𝐫`
            Δ𝐫 = zeros(3)
            Δ𝐫[i] = δ  # Set the `i`th component of `Δ𝐫` to be `δ`
            u₁ = potential_energy(𝐫 + Δ𝐫)  # New potential energy between the two particles
            @test (u₁ - u₀) / δ - potential_gradient(𝐫)[i] < 1e-8
        end
    \end{juliacode}
\end{algorithm}
%
The negative sign indicates the force is the negative gradient of the potential.
To get the total potential energy of the swarm of particles, we sum up all the $i$'s
and $j$'s (except for $j = i$) in~\eqref{eq:U}:
%
\begin{equation}\label{eq:U}
    U = \frac{ 1 }{ 2 }\sum_{\substack{i, j\\ j \neq i}} u(r_{ij}).
\end{equation}
%
Note the $\frac{ 1 }{ 2 }$ factor in \eqref{eq:U} since we count $i$ and $j$ twice.
Snippet \ref{lst:U} does the same thing:

\begin{algorithm}
    \caption{Calculate the total Lennard--Jones potential energy of a swarm of particles.}
    \label{lst:U}
    \begin{juliacode}
        function potential_energy(particles::Vector{Particle})
            return 1 / 2 * sum(eachindex(particles)) do i
                sum(filter(!=(i), eachindex(particles))) do j
                    potential_energy(particles[i], particles[j])
                end
            end
        end
    \end{juliacode}
\end{algorithm}

Similarly, we need to calculate the total force of all the other particles
exert on one particle, as shown in \eqref{eq:flj}. Now we need to employ the
\code{find_neighbors} function we used to calculate one particle's neighbors and plotted
Figure \ref{fig:neighbors}, as shown in \ref{lst:flj}.
%
\begin{algorithm}
    \caption{Calculate the Lennard--Jones force a the particle feels that are exerted by
        the rest of the particles.}
    \label{lst:flj}
    \begin{juliacode}
        function force(particle::Particle, particles, box::Box)
            neighbors = find_neighbors(particle, particles, box)
            return sum(Force(particle), neighbors)
        end
    \end{juliacode}
\end{algorithm}
